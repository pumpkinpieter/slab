\documentclass[12pt,a4paper]{article}
\usepackage[utf8]{inputenc}
\usepackage[english]{babel}
\usepackage{amsmath}
\usepackage{mathrsfs}
\usepackage{amsfonts}
\usepackage{amssymb}
\usepackage{amsthm}
\usepackage{graphicx}
\usepackage{cancel}
\usepackage{subcaption}
\usepackage{cleveref}
\usepackage[left=1.5cm,right=1.5cm,top=2cm,bottom=2cm]{geometry}

\theoremstyle{definition}
\newtheorem{example}{Example}[section]
\newtheorem{theorem}{Theorem}[section]

\newcommand{\lp}{\left(}
\newcommand{\rp}{\right)}
\newcommand{\goestoposinf}{\rightarrow\infty}
\author{Pieter Vandenberge}
\title{Method of Steepest Descent}
\begin{document}
\maketitle
These are notes to better understand the method of steepest descent, which provides asymptotic approximations to functions of the form 
%
\begin{align}\label{eq:main function standard form}
I(\Omega) = \int_\mathcal{C} f(z)e^{\Omega g(z)}  dz
\end{align}
%
for large values of $\Omega$.  Here $\mathcal{C}$ is a contour in the complex plane.  The notation chosen here is the same as that found in  \cite{jacksonSteepestDescentMethod}.

\section{Laplace's Method}

In the case that the contour is a portion of the real axis, the method reduces the Laplace's method and the above function can be written as
%
\begin{align}\label{eq:main function laplace form}
I(\Omega) = \int_a^b f(z)e^{\Omega g(z)}  dz.
\end{align}
%
This method works well if $g(z)$ has a convergent power series at some $z_0$ in the interval $[a,b]$ at which it attains a local maximum.  In this case
%
\begin{align}
g(z) - g(z_0) & = \cancelto{0}{g'(z_0)} (z - z_0) + \frac{g''(z_0)}{2} (z-z_0)^2 + \mathcal{O} \left( (z-z_0)^3 \right)\\[.05in]
 & = -\frac{|g''(z_0)|}{2} (z-z_0)^2 + \mathcal{O} \left( (z-z_0)^3\right) \label{eq:g minus g0}
\end{align}
%
where we have used the fact that $g''(z_0) < 0$.  Re-writing \cref{eq:main function laplace form} using \cref{eq:g minus g0} we obtain
%
\begin{align}
I(\Omega) & = \int_a^b f(z)e^{\Omega g(z)}  dz \\[.05in]
 & = e^{-\Omega g(z_0)}  \int_a^b f(z)e^{\Omega \lp g(z) - g(z_0)\rp} dz \\[.05in]
 & = e^{-\Omega g(z_0)}  \int_a^b f(z)e^{-\Omega \lp \frac{|g''(z_0)|}{2} (z-z_0)^2 - \mathcal{O} \left( (z-z_0)^3\right)\rp} dz.
\end{align}

The motivation for Laplace's method comes from the following observations:

\begin{itemize}
\item  The exponential term in the above integral is essentially of the form  $e^{-\Omega s^2}$, and the dominant portion of the integral comes from integrating this Gaussian function near $s = z_0$.

\item This behavior will increase as $\Omega \goestoposinf$. 
\end{itemize}

\begin{example}\label{ex:cos}
Let 
$$ I(\Omega) = \int_{-\pi}^\pi \cos(z)e^{-\Omega z^2}  dz.$$

For increasing values of $\Omega$, the integrand appears as in \cref{fig:cos example integrand}.

\begin{figure}[htbp]
\centering
\includegraphics[width=.9\textwidth]{cos_example_integrand.jpg}
\caption{Behavior of integrand from \cref{ex:cos}}\label{fig:cos example integrand}
\end{figure}

Hence, for large $\Omega$, the integral will largely be determined by $f(z_0)$ and the value of the Gaussian integral on that interval.  Furthermore, as $\Omega$ increases the Guassian integral's limits may be extended to $\pm \infty$ without affecting the value.  Hence, for large $\Omega$, we should have
%
\begin{align*}
 I(\Omega) & \sim \cos(0) \int_{-\infty}^\infty e^{-\Omega z^2}  dz \\
 & = \sqrt{\frac{\pi}{\Omega}}.
\end{align*}
\end{example}

As shown in \cref{fig:cos example asymptotics}, this approximation yields a very good estimate of true value of the function very quickly as $\Omega$ increases. Unfortunately, another example fails to give useful results.
%
\begin{figure}[htbp]
\centering
\includegraphics[width=.9\textwidth]{cos_example_asymptotics.jpg}
\caption{Asymptotics of $I(\Omega)$ from \cref{ex:cos}}\label{fig:cos example asymptotics}
\end{figure}

\begin{example}
Consider the function
%
$$ I(\Omega) = \int_{0}^\infty z e^{-\Omega z^2}  dz.$$
%
Were we to repeat the argument of \cref{ex:cos}, we would determine that
%
\begin{align*}
 I(\Omega) & \sim 0 \int_{0}^\infty e^{-\Omega z^2}  dz \\
 & = 0
\end{align*}
%
for large values of $\Omega$.  While this may be technically true, it is not a helpful estimate of the function values as we increase $\Omega$, nor does it tell us anything about the rate at which the values decay to zero.

On the other hand, this example has a closed form solution, namely:
%
$$ I(\Omega) = \frac{1}{2\Omega},$$
%
which shows that the true solution does decay to zero, and additionally shows us the rate of decay.  We hope  that a more fully developed version of Laplace's method will allow us to determine this behavior.
\end{example}
Now more









\newpage
\bibliography{../../slab.bib}
\bibliographystyle{siam}



\end{document}